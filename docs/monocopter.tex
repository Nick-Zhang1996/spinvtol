\documentclass[12pt]{article}
\usepackage{pdfpages}
\usepackage{graphicx}
\usepackage{float}
\usepackage{url}
\usepackage{makecell}
\newcommand{\pa}[1]{\paragraph{#1}\mbox{}\\}

\title{Monocopter Manual}
\author{Zhiyuan Zhang}
\date{\today}

\begin{document}
\maketitle
\section{Summary}
This document aims to provide detailed instructions on reproducing all the design and fabrication process of a monocopter platform worked on by Zhiyuan Zhang during his internship at Computational Fabrication Group at MIT CSAIL in summer 2019.

For any questions or comments, contact Zhiyuan Zhang at nickzhang@gatech.edu. As a general guideline, should you encounter any problems, contact me ASAP as that may save you a lot of time. Your mental cost for disturbing me should be considered zero. 

\section{Sizing Analysis}
A typical design for a monocopter starts with rough sizing. 

A takeoff weight is estimated from available propulsion system, payload, and structural layout of the monocopter. Once a weight is decided, blade element theory is utilized for determing the appropriate span and aspect ratio for the monocopter. Many iterations may be required to obtain a design with good desired properties.

Some factors to consider include:
\begin{itemize}
  \item Structural strength, which eliminates the use of high aspect ratio and very thin airfoils.
  \item Rotational Speed, upper bounds for rotational speed include structural strength, electronics' capabilities, maximum available propulsion, etc. Lower bounds include disk angle (discussed further in Tuning section), stability, etc.
  \item Aerodynamic restrains, the $C_l$ must be reasonable for the airfoil chosen.
  \item Tip velocity, too high tip velocity would be dangerous to pilot should a part comes loose.
\end{itemize}

A small script \texttt{analysis.py} calculates required $C_l$, disk loading, tip acceleration, tip velocity and some other specs from wing size, gross weight, and rotational speed.

Current monocopter weights 500g and has a 11.7\% thickness Clark-Y airfoiled wing spanning 60cm with a cordlength of 19cm.
Depending on the installation angle of the wing (-2 ~ 12 deg), the monocopter can hover with rotational speed ranging from 3.5 ~ 6 rev/s. 


\subsection{Adjusting Miscelleous Components}
NOTE: This section is based on crude observation and does not contain any proof or math to support it.

Many factors contibute to a successful flight, this section discusses some general instructions on adjusting a monocopter and some ``gotcha'' s in configuring the monocopter.

For ease of discussion, define a right-hand cartesian coordinate system with origin coincident with vehicle center of mass, X axis parallel to the carbon rod, pointing from the trailing edge to leading edge, Y axis parallel with wing span, pointing outward. Z axis points downward.

Stable hovering requires the rotational axis should be the first or third principle axis of the monocopter, this is generally maintained with the default configuration (first principle axis), however, if significant modification has been made, this may need additional attention.

The three moments are of particular importance for stable hovering of the monocopter. Forces mostly balance out with the rotation of the monocopter.

Moment about the Z axis is straightforward, the thrust of the motor counteracts all drag generated by the rotational movement of the monocopter. A higher thrust result in a higher rotational velocity.

Movement about Y axis is subtle and mostly small periodic oscillations as a result of control input. Actuation of the flap alters the pitching moment from the wing, and result in a small cyclic variation in pitching angle, this effect is undesirable and controlled by $I_{yy}$ Increasing the distance between the motor and electronics plate effectively increases $I_{yy}$. 
One source of Y axis moment is the torque from motor if the motor is not aligned with Y axis. In current configuration the motor should be parallel to the y axis. The drag from spinning the propeller also contribute to Y axis momoent.

On some rare configurations the wing would continusly pitch up, and y axis would point towards sky rapidly, resulting in a violent crash. This can be corrected by either shortening the wing or adding a 20 gram counterweight to the outmost wing section. The cause of this issue is unknown due to the author's incompetence. 

Moment around X axis is of crucial importance to flight characteristics. The lift from the wing generates a positive moment, this results in a positive disk angle, defined as the angle between the rotational plane and y axis of the monocopter. The faster the monocopter rotates, the smaller this angle will be. A higher disk angle can be beneficial to stability of the monocopter, whereas a smaller disk angle makes the monocopter easier to maneuver. A resonable rotational speed found in expertiment is 5 rev/s, although a speed as low as 3.5 has supported successful flight, such low speed typically result in large (>15 deg) disk angle and seems close to the boundary of the performance envelope. Another source of moment around X axis comes from the rotational inertia of the motor's spinner and propeller. The rotation direction of the motor is chose so it reduces disk angle during flight. 

\pa{Adjust Installation Angle of Wing}
Disk angle can be lowered by reducing lift coefficient of the wing, this is mainly done by adjusting the installation angle of the main wing. Loosen all four screws on the wing assembly, user the indicator on the rear upper slider to select between -2 to +12 degree of installation angle. 

Table below lists some experimental data between disk angle and various flight parameters for reference.\footnote{Average Cl is calculated using the script mentioned in previous section. AOA based on Re=2e5. Thrust are standing pull, actual thrust in flight will be lower. Disk angle measurements have around 5 deg measurement error, Omega is obtained from slow motion camera on IOS using CMV Free App}

\begin{table}[]
  \begin{tabular}{|c|c|c|c|c|c|}
    \hline 
    \makecell{Install Angle \\ (deg)} & \makecell{Omega \\ (rev/s)} & \makecell{Thrust \\ (N)} & \makecell{Disk Angle \\ (deg)} & \makecell{Average \\ lift coeff} & \makecell{Theoretical \\ AOA(deg)} \\\hline
    3                  & 5.0           & 3.36      & 10              & 0.63         & 1.9                  \\\hline
    5                  & 3.85          & 3.18      & 15              & 1.06         & 6.5                  \\\hline
    7                  & 3.57          & 2.9       & 15              & 1.24         & 8.6                 \\\hline
  \end{tabular}
\end{table}

For the monocopter to take off from a standing still position on floor, it must be configured in the right orientation so it can generate lift while remaining contact with ground. 
The nuts on the landing gear assembly below the motor stand can be adjusted to modify the incident angle of the wing while the monocopter is constrained to the floor. It's a good idea to set the carbon fiber rod parallel to the ground so the wing has the same incident angle as its installation angle.

Once the incident angle is configured, adjust the electronics plate so the wing stays horizontal. The weight of the monocopter should rest on the front landing ring, the rear mousewheel, and the block of foam attached to the end of the electronics plate. The angle between electronics plate and the rest of the assembly is important. 

The center of gravity should be set at around quarter chord, this would minimize the variation in pitching moment from the wing as a result of varying angle of attack.

\section{Mechanical Components}
This sections describes details of the mechanical components on the monocopter and means of fabrication.
\subsection{Propulsion}
The monocopter is powered by GEMFAN 6042 3 blade propeller coupled to a EMAX MT2204 2300Kv brushless motor, 3s Lipo, and a 12A BLHeli ESC(with BEC). The propeller is a counter-clockwise propeller (omega and thrust same direction), installed reversely to a clockwise motor so the angular velocity vector and thrust vector points toward the base of the motor.  Replacement parts can be easily ordered online. An additional brushless motor of same spec is stocked, but the threading is intended for a different spinning direction. CCW and CW motor should be identical except for right/left hand thread in spinner shaped nut used to hold the propeller down. The nut would tend to loose from use if the motor is used in the ``wrong'' direction, but some locktite should fix this. 

\subsection{Wing}
The wing is made from standard pink insulation purchased from local home depot, using a CNC hotwire cutter from MIT AeroAstro department. The machine is intended for sole use by MIT AeroAstro students, but not really restricted to others. Go to building 33, 1st floor, you should find a large space with model aircraft hanging on the ceiling, go downstairs into the space with many tables(the door in the basement downstairs is usually locked, but the door on the 1st floor is open during normal business hours), keep walking with tables on your left and a machine shop on your right, turn right into a room full of pink insulation foams, there's a computer and the CNC hotwire machine. The computer has on its desktop a detailed video on how to use everything. What you need to bring are the foam and airfoil files in dat format, generated by xfoil. The airfoil used by this monocopter is clark-y and can be found in the project repository. You need to start the control box below the computer, open the GUI for foam cutter, load the dat file as xfoil data (which allows you to customize AOA and chord length independenly for left and right, though we use the same for a simple rectangular wing), then align the axes with move command. Finally place the foam, weight, then turn on the power supply for the wire and adjust it for 2Amp current, then click run and enjoy the fumes of cancer(facemask with low evaporation point organic fume protection(the pink canister from 3M) is recommended). Everything is covered in the video and should be pretty straight forward. We have some extra stock foam in stock, if more is needed they can be easily bought from Home Depot, but they only come in very large size (for the proper thickness), and may require cutting if you don't drive an F450.

Should you decide to make a new wing, look closely at the previous wings for reference.

Wing need to be attached to wing installation plate using expoxy and fiber tape. Solidworks files and dxf files can be found in the project GrabCAD repository. Recommended workflow for attaching wing to the installation plate is 
\begin{enumerate}
  \item Recut wing in the chord direction to ensure the plane is orthogonal to span, then sand the cross section carefully, avoid excessive pressure on the belt sander to avoid burning of the foam. 
  \item Sand the installation plate for a rougher surface
  \item Apply epoxy to a surface
  \item Using a long strip of fiber tape, tape the installation plate tightly to the wing. The tape must be tight and long to take load, especially on the lower surface.
  \item Allow epoxy to dry for 24 hours
\end{enumerate}

The wing should be 0.6m spanwise and 0.2m chordwise according to current design. 

To add a flap control surface:
\begin{enumerate}
  \item Draw a 20cm by 6cm rectangle near the outside trailing edge, cut it carefully with dense bandsaw blade.
  \item Sand both surfaces, make sure there is a draft on the wing that leaves more material on the top. The draft would allow free movement of the flap without interference. The wing should be connected with the control surface by a line. 
  \item Tape the upper surface together with fiber tape
  \item Deflect the control surface FULLY upward, in this position, tape the lower surfaces with fiber tape, make sure the tape goes all the way into the corners, ideally just touching the tape on the upper surface. If the tape is not tight, the flap would not be able to go upwards.
  \item Dig out a slot in the middle across the span of the control surface, epoxy a control tab to it. Control tab can be home made or found in the leftovers from Jie's practice RC airplane kits.
  \item Dig out a hole in exact shape of the servo ( we use a very fast KT servo). The size should be slightly smaller than the servo to allow a snug fit once the servo is presed in. Choose the location carefully, make sure you have enough thickness to seat the servo, and minimize the connection rod length to avoid buckling
  \item Keep the removed portion, sand it down and put it on top of the servo, tape it down. The material you remove from the wing should be exactly the same size as the servo, the servo should be able to rest snugly in the cavity.
  \item Find a very thin carbon steel rod in the leftovers of Jie's project's practice RC airplanes. Bend it to the right shape and length.
  \item First attach the rod to the control tab and servo control arm, then install the servo, if the servo is not tightly secured in place, hot glue it down.
  \item Test the servo's movement, determine proper range of the servo. Note the servo se have uses 300Hz short pulse PPM (760us for neutral point). Standard servo tester cannot generate such short pulse. In the Arduino folder in the project's repository there is a program called shortPulseServoTester.ino that maps standard control pulse to the signal our special servo requires. Use that. There shouldn't be much difference. The center of the servo should be at 745us, which should translate to 0 degree deflection of the flap. If not off by a lot (<5 deg), this can easily be adjusted at flight time with the transmitter.
\end{enumerate}

Wing may be repaired using epoxy(maybe hotglue) and fiber tape. Epoxy restores internal structural integrity while fiber tape reduces the negative effect of stress concentrations near the surface. Fiber tape must be applied tightly so it takes load before the foam deforms much. The fiber tape should be applied to bothsides of the wing, and extend to the wing root whenever possible. 

You should not need to do everything above, we have several wings in stock, mostly in usable conditions. 

\subsection{Other Components}


\section{Software and Electronics}
\subsection{Arduino based system}
\subsubsection{Hardware}
\subsubsection{Software}

\subsection{Teensy based system}
\subsubsection{Hardware}
\subsubsection{Software}

\end{document}
