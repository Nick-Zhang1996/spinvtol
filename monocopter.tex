\documentclass[12pt]{article}
\usepackage{pdfpages}
\usepackage{graphicx}
\usepackage{float}
\usepackage{url}
\newcommand{\pa}[1]{\paragraph{#1}\mbox{}\\}

\title{Monocopter Manual}
\author{Zhiyuan Zhang}
\date{\today}

\begin{document}
\maketitle
\section{Summary}
This document aims to provide detailed instructions on reproducing all the design and fabrication process of a monocopter platform worked on by Zhiyuan Zhang during his internship at Computational Fabrication Group at MIT CSAIL in summer 2019.

For any questions or comments, contact Zhiyuan Zhang at nickzhang@gatech.edu

\section{Sizing Analysis}
A typical design for a monocopter starts with rough sizing. 

A takeoff weight is estimated from available propulsion system, payload, and structural layout of the monocopter. Once a weight is decided, blade element theory is utilized for determing the appropriate span and aspect ratio for the monocopter. Many iterations may be required to obtain a design with good desired properties.

Some factors to consider include:
\begin{itemize}
  \item Structural strength, which eliminates the use of high aspect ratio and very thin airfoils.
  \item Rotational Speed, upper bounds for rotational speed include structural strength, electronics' capabilities, maximum available propulsion, etc. Lower bounds include disk angle (discussed further in Tuning section), stability, etc.
  \item Aerodynamic restrains, the $C_l$ must be reasonable for the airfoil chosen.
  \item Tip velocity, too high tip velocity would be dangerous to pilot should a part comes loose.
\end{itemize}

A small script \texttt{analysis.py} calculates required $C_l$, disk loading, tip acceleration, tip velocity and some other specs from wing size, gross weight, and rotational speed.

Current monocopter weights 500g and has a 11.7\% thickness Clark-Y airfoiled wing spanning 60cm with a cordlength of 19cm.
Depending on the installation angle of the wing (-2 ~ 12 deg), the monocopter can hover with rotational speed ranging from 3.5 ~ 6 rev/s. 


\subsection{Force Balance}
For ease of discussion, define a right-hand cartesian coordinate system with original coincident with vehicle center of mass, X axis parallel to the carbon rod, pointing from the trailing edge to leading edge, Y axis parallel with wing span, pointing outward. Z axis points downward.

Three moments are of particular importance for stable hovering of the monocopter. Forces mostly balance out with the rotation of the monocopter.

First, moment around X axis. The lift from the wing generates a positive moment, balanced by 

\section{Mechanical Components}
This sections describes details of the mechanical components on the monocopter and means of fabrication.
\subsection{Propulsion}


\subsection{wing}
\subsection{Tuning}

\section{Software and Electronics}
\subsection{Arduino based system}
\subsubsection{Hardware}
\subsubsection{Software}

\subsection{Teensy based system}
\subsubsection{Hardware}
\subsubsection{Software}

\end{document}
